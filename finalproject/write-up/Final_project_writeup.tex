\documentclass[11pt, oneside]{article}   	% use "amsart" instead of "article" for AMSLaTeX format
\usepackage{geometry}                		% See geometry.pdf to learn the layout options. There are lots.
\usepackage{textcomp}
\usepackage{hyperref}  % TODO: see page 94 of latex book
\geometry{letterpaper}                   		% ... or a4paper or a5paper or ... 
%\usepackage[parfill]{parskip}    		% Activate to begin paragraphs with an empty line rather than an indent
\usepackage{graphicx}				% Use pdf, png, jpg, or eps§ with pdflatex; use eps in DVI mode
								% TeX will automatically convert eps --> pdf in pdflatex		
\usepackage{amssymb}
\usepackage{amsmath}
\usepackage{relsize}
\usepackage{parskip}


\title{CS181 / CSCI E-181 Spring 2014 Final Project}
\author{
  David Wihl\\
  \texttt{davidwihl@gmail.com}
  \and
  Zachary Hendlin\\
  \texttt{zgh@mit.edu} 
}
%\date{}							% Activate to display a given date or no date


\begin{document}
\maketitle
\section*{Introduction}
To gather sufficient data, we allowed the SampleAgent (provided intially in the code) to play ~30 megabytes of ghost training data of games, to get data on feature vectors and associated point values of each.

\section*{Classification of Ghosts}
We sought to classify ghosts as [0, 1, 2, 3, 5], where all ghosts in category 5 are dangerous (e.g. induce a reward of -1000 points unless a helpful capsule is consumed first).

We explored two methods for classifying the ghosts on the basis of their features.

First, we explored linear support vector machines using SK-Learn's Stochastic Gradient Descent classifier.
For classifying the category 5 ghosts, this approach was accurate 90.62 percent of the time. Our analysis found that while differences in the rewards associated with eating ghosts not from class 5 did differ by class, the most important thing for us to measure our performance on is the correct classification of dangerous (class 5 ghosts).

<graph here>

We also used logistic regression classification, and achieved somewhat better results, with correct classification of class 5 ghosts 93.25 percent of the time.

We elected to use the logistic regression classification results to predict which class each ghost is in during runtime.

\section*{Classification of Placebos}
\section*{Reinforcement Learning}
\section*{Conclusion}


\begin{thebibliography}{1}

 \bibitem{item1}\emph{Reinforcement Learning}, Sutton \& Barto, 1998, ISBN-10: 0-262-19398-1
 
  \end{thebibliography}

\end{document}  
